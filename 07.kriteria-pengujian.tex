\chapter{KRITERIA PENGUJIAN PROGRAM}

Kriteria yang diperlukan untuk melakukan pengujian kesesuaian program terhadap spesifikasi, karena akan menggunakan \textit{unit test} dan \textit{integration test}, maka beberapa poin kriterianya adalah seperti berikut ini :

\begin{itemize}
	\item Pengujian terhadap data objek pajak :
	
	\begin{itemize}
		\item Nomor objek pajak yang dikembalikan dari sistem basis data harus sesuai seperti data yang dikirimkan saat klien melakukan \textit{request}.
		\item Luas bumi yang dihasilkan berdasarkan Nomor Objek Pajak tertentu harus sama besarnya seperti yang tertera dalam sistem basis data.
		\item Luas bangunan yang dihasilkan berdasarkan Nomor Objek Pajak tertentu harus sama besarnya seperti yang tertera dalam sistem basis data.
		\item Nilai jual objek pajak bumi yang dihasilkan berdasarkan Nomor Objek Pajak tertentu harus sama besarnya seperti yang tertera dalam sistem basis data.
		\item Nilai jual objek pajak bangunan yang dihasilkan berdasarkan Nomor Objek Pajak tertentu harus sama besarnya seperti yang tertera dalam sistem basis data.
		\item Nama jalan yang dihasilkan berdasarkan Nomor Objek Pajak tertentu harus sama nilainya seperti yang tertera dalam sistem basis data.
		\item Nomor blok, atau nomor rumah yang dihasilkan berdasarkan Nomor Objek Pajak harus sama nilainya seperti yang tertera dalam sistem basis data.
		\item Nomor RT yang dihasilkan berdasarkan Nomor Objek Pajak tertentu harus sama nilainya seperti yang tertera dalam sistem basis data.
		\item Nomor RW yang dihasilkan berdasarkan Nomor Objek Pajak tertentu harus sama nilainya seperti yang tertera dalam sistem basis data.
		\item Nama Kecamatan yang dihasilkan berdasarkan Nomor Objek Pajak tertentu harus sama seperti yang tertera dalam sistem basis data.
		\item Nama Kelurahan / Desa yang dihasilkan berdasarkan Nomor Objek Pajak tertentu harus sama seperti yang tertera dalam sistem basis data.
	\end{itemize}
	
	\item Pengujian terhadap data subjek pajak :
	
	\begin{itemize}
		\item Nama subjek pajak yang dihasilkan berdasarkan Nomor Objek Pajak tertentu, hasilnya harus sama seperti yang tertera dalam sistem basis data.
		\item Nama jalan tempat subjek pajak tinggal yang dihasilkan berdasarkan Nomor Objek Pajak tertentu, hasilnya harus sama seperti yang tertera dalam sistem basis data.
		\item Nomor blok atau nomor rumah tempat subjek pajak tinggal yang dihasilkan berdasarkan Nomor Objek Pajak tertentu, hasilnya harus sama seperti yang tertera dalam sistem basis data.
		\item Nomor RT tempat subjek pajak tinggal yang dihasilkan berdasarkan Nomor Objek Pajak tertentu, hasilnya harus sama seperti yang tertera dalam sistem basis data.
		\item Nomor RW tempat subjek pajak tinggal yang dihasilkan berdasarkan Nomor Objek Pajak tertentu, harus sama seperti yang tertera dalam sistem basis data.
		\item Nama Kelurahan / Desa tempat subjek pajak tinggal berdasarkan Nomor objek Pajak tertentu, isinya harus sama seperti yang tertera dalam sistem basis data.
		\item Nama Kota tempat subjek pajak tinggal berdasarkan Nomor Objek Pajak tertentu, isinya harus sama seperti yang tertera dalam sistem basis data.
	\end{itemize}
	
	\item Pengujian terhadap data tagihan dari Surat Pemberitahuan Pajak Terhutang (SPPT) untuk tiap tahun pajak : 
	
	\begin{itemize}
		\item Besarnya pajak terhutang untuk setiap tahun pajak berdasarkan Nomor objek Pajak tertentu, harus sama nilanya seperti kondisi pada sistem basis data.
		\item Kondisi status pembayaran untuk setiap tahun pajak berdasarkan Nomor Objek Pajak tertentu, harus sama nilanya seperti kondisi pada sistem basis data.
	\end{itemize}
\end{itemize}
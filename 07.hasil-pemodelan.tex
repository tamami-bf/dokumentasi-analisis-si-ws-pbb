\chapter{HASIL PEMODELAN}

Hasil pemodelan dari sistem aplikasi yang akan dibangun, secara internal sebetulnya sudah dibahas pada bagian Arsitektur Sistem dan diperjelas dengan penjelasan pada bagian Deskripsi Sub Sistem. 

Secara tampilan tatap muka (\textit{interface}) tidak akan menghasilkan apa-apa selain model dari format \textit{request} yang akan menjadi 3 (tiga) bagian berikut :

\begin{enumerate}
  \item \texttt{pospbb/sppt/\{nop\}/\{thn\}}
  
  \textit{Request} ini digunakan untuk \textit{inquiry} data tagihan PBB-P2, dimana \texttt{nop} nantinya digantikan dengan nomor objek pajak PBB-P2 tanpa tanda baca, dan \texttt{thn} adalah tahun pajak dimana NOP tersebut akan dilihat informasinya.
  
  \item \texttt{pospbb/bayar/\{nop\}/\{thn\}/\{tglBayar\}/\{jamBayar\}}
  
  \textit{Request} ini digunakan untuk melakukan pencatatan pembayaran, dimana \texttt{nop} adalah Nomor Objek Pajak yang dibayarkan, \texttt{thn} adalah tahun pajak yang dibayarkan, \texttt{tglBayar} adalah tanggal terjadinya pembayaran dalam format DDMMYYYY dimana DD adalah 2 (dua) digit tanggal dalam satu bulan, MM adalah 2 (dua) digit bulan, dan YYYY adalah tahun. \texttt{jamBayar} adalah jam dibayarkannya PBB-P2 dengan format HH24MI, dengan HH24 dimaksudkan adalah 2 (dua) digit jam dengan format 24 jam, dan MI adalah menitnya.
  
  \item \texttt{pospbb/reversal/\{nop\}/\{thn\}/\{ntpd\}}
  
  \textit{Request} ini adalah \textit{request} untuk melakukan \textit{reversal} data pembayaran, dimana \texttt{nop} adalah Nomor Objek Pajak yang akan dilakukan \textit{reversal}, \texttt{thn} adalah Tahun pajak untuk NOP yang akan dilakukan \textit{reversal}, sedangkan \texttt{ntpd} adalah Nomor Transaksi Penerimaan Daerah sebagai identitas pencatatan pembayaran yang telah terjadi.
  
\end{enumerate}

Ini adalah format standar yang nantinya digunakan untuk berkomunikasi dengan \textit{client} dalam hal ini pihak Bank sebagai tempat pembayaran dalam hal pencatatan transaksi pembayaran.